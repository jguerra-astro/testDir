\documentclass{article}
\usepackage[utf8]{inputenc}
\usepackage{mathptmx}
\usepackage{amsmath} % AMS Math Package
\usepackage{amsthm} % Theorem Formatting
\usepackage[letterpaper,margin=1in]{geometry} %changes the margin----
\usepackage{commath}
\usepackage{enumitem}
\usepackage{siunitx}
\usepackage{parskip}

\newtheorem{p}{}[section]
\title{ASTR 500: The Physics of Astrophysics}
\author{Juan Guerra}
\date{Fall 2018}
\begin{document}

\maketitle

\section{Problem Set 1}
	
	\textbf{1. Apparent magnitude of a supernova}
	
    	The absolute magnitude of a start in the galaxy M81 (distance 3.6 Mpc) is $M=6.$ It explodes as a supernova, and becomes $10^{9.5}$ times brighter. What is the apparent magnitude of the supernova?
	
    \textit{answer:}
    
        relationship between relative magnitude of two objects:
        
        $m_{1}-m_{2}=2.5 \, log_{10}(\frac{l_{1}}{l_{2}})$
        
        relation between apparent magnitude and absolute magnitude:
        
        $m-M = 5\, log_{10}\frac{d}{10}$
        
        find the apparent magnitude of the star before its a super nova
        
        $m_{1}=6+5\,log_{10}(\frac{3.6 x 10^{6}}{10})\hfill m_{1}=33.7815$
        
        apparent magnitude of star once its  a supernova
       
        $m_{2}= m_{1}-\textbf{} 2.5\, log_{10}(\frac{10^{9.5}}{1})\hfill \textbf{}m_{2}= 10.03$
	
	
	\textbf{2. Fourier Transform of the step function}

		Find the Fourier transform  of the function $f(t)$ define as:
		\begin{equation}
		  f(t) =
		  \begin{cases}
		   1 & \abs{t} <1 \\
		   0 & \text{otherwise} \\
		  \end{cases}
		\end{equation}
	

	
		\textbf{Fourier Transform of irregularly sampled data (5 pts)}
		
		The time series of energy emitted by a variable astronomical source is give by:
		\begin{equation}
			f(t)=a_{1}cos(\omega_{1}t -\delta_{1})
		\end{equation}
		Find the Fourier transform of this function if the signal is detected with gaps. That is, it is observed from time $t=0$ to $t=T$ and then re-observed after 3 nights (after an elapsed time $t_{0}$) from $t_{0}$ to $t_{0}+T$ Plot the power spectrum of this source. Also show how the power spectrum changes as the sampling gap $t_{0}$ is increased to $1.5T$ and $2T.$

	

	
	\textbf{3. Radiation from nearly isotropic sources}

		When the radiation field is very nearly isotropic, we can expand the intensity $I(\theta)$ in a Taylor series about $\theta =0$ as follows: 
		\begin{equation}
			I(\theta)= I_{0} + I_{1} cos(\theta) + ...
		\end{equation}
		Where $I_{0}$ and $I_{1}$ are independent of $\theta$. Relate these two coefficients to the flux and energy density of the field. Under what conditions on the energy density and flux does the following relation hold:
		\begin{equation}
			\abs{F^{in}_{\nu}} = \abs{F^{out}_{\nu}} = \pi I_{\nu}
		\end{equation}
	

	
	\textbf{4. Radiation from nearly isotropic sources}

		When the radiation field is very nearly isotropic, we can expand the intensity $I(\theta)$ in a Taylor series about $\theta =0$ as follows: 
		\begin{equation}
			I(\theta)= I_{0} + I_{1} cos(\theta) + ...
		\end{equation}
		Where $I_{0}$ and $I_{1}$ are independent of $\theta$. Relate these two coefficients to the flux and energy density of the field. Under what conditions on the energy density and flux does the following relation hold:
		\begin{equation}
			\abs{F^{in}_{\nu}} = \abs{F^{out}_{\nu}} = \pi I_{\nu}
		\end{equation}
	
	\textbf{5. Energy density of radiation from a central source}
		\begin{enumerate}[label=(\alph*)]
			\item Show that the mean intensity at a distance r from a uniformly bright sphere of radius $R$ is give by:
			\begin{equation}
				J_{\nu}=\frac{1}{2} I_{\nu} \Bigg[1 - \sqrt{1-\frac{R^{2}}{r^{2}}}\Bigg].
			\end{equation}
			\item Show that surface brightness $I_{\nu}$ of the sphere is given in terms of its monochromatic luminosity by:
			\begin{equation}
				I_{\nu}= \frac{L_{\nu}}{4\pi^{2}R^{2}}.
			\end{equation}
			\item Hence show that the total (i.e. frequency integrated) energy density of radiation at a distance r from a uniformly emitting sphere, where r is large compared to its radius, is given by: 
			\begin{equation}
				u=\frac{L}{4\pi c r^2}.
			\end{equation}
			Where $L$ is the luminosity. Note that $u$ is independent of the radius of the sphere. 
			\item  Estimate the energy density of sunlight at the Earths orbit but outside the Earths atmosphere. You will need the following parameters:
			\begin{itemize}
				\item Luminosity of the Sun: \SI{3.83e26}{\watt}
				\item Sun - Earth distance: \SI{1.5e11}{\meter}
			\end{itemize}
			Express your answer both in units of $J m^{-3}$ and in $eV m^{-3}$.
		\end{enumerate}
	

	
	\textbf{6. Eddington Atmospheres}

		\noindent
		Show that the solution to the radiative transfer equation under the Eddington approximation is no self-consistent, i.e. compare the expressions for $J$ in terms of the flux calculated at the surface with and without making the approximation. 
	
\newpage %-------------------------------------------------------------------------- 
\section{Problem Set 2}
	
	\textbf{1. Thermal Broadening}
	
		Calculate the thermal broadening of a line of wavelength 656 nm in  a hydrogen plasma at $10^8 K$. This is the typical temperature in the inner region of a cluster of galaxies- int intra-cluster medium - that glows in the X-rays. 
	
    \textbf{2. Relation between Einstein Co-efficients}

		Derive the relation between the Einstein coefficients and clearly describe the conditions under which they hold.
	

	
	\textbf{3. Relation between Einstein Co-efficients}
		
		Derive the relation between the Einstein coefficients and clearly describe the conditions under which they hold.
	

	
	\textbf{4. The Saha equation}

		Using the Saha equation determine the degree of ionization of hydrogen in the center of the center of the Sun where the density $\rho = 100 g cm^3$ and the temperature $T=$\SI{1.5e7}{\kelvin}. 
	

	
	\textbf{5. Differential equations}

		Solve the following differential equations using the standard methods.
		\begin{equation}
			5y\frac{dy}{dx} +4x =0
		\end{equation}
		\begin{equation}
			\frac{dy}{dx}= sec(y)sec(x)
		\end{equation}
		\begin{equation}
			2xy\frac{dy}{dx} =y^{2}-x^{2} =0
		\end{equation}
		\begin{equation}
			xdy-ydx=\sqrt{x^{2}+y^{2}} dx
		\end{equation}
		\begin{equation}
			\frac{dy}{dx} -2y =e^{2x}
		\end{equation}
		\begin{equation}
			\frac{d^2y}{dx^2} +3\frac{dy}{dx} - 13y =0
		\end{equation}

	

	
	\textbf{6. Stellar Model}

		For a physics model of a star of mass M and radius R where the density $\rho(r)$ is a linear function of radius with constant density $\rho_{0}$ at the center and $\rho=0$ at $r=R$, derive the radial dependence of the mass. What fraction of the total mass is enclosed at half the radius, where $r=\frac{r}{2}$
	


\end{document}
